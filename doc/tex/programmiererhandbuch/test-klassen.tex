\subsection{Test.gui Quellpaket}
    \subsubsection{FakeGUI}
        \begin{table}[H]
            \caption{Klasse FakeGUI}
            \begin{tabular}{p{2.5cm}  p{9.5cm}} 
                \hline
                \textbf{Eigenschaft} & \textbf{Beschreibung}\\
                \hline
                Name & FakeGUI\\
                Ort & Quellpaket \textit{test.gui}\\
                \hline
                Zweck &
                Eine Klasse die das \textit{GUIConnector}-Interface implementiert. Dabei sind die Methoden jedoch so implementiert,
                dass diese nichts tun. So können Tests ausgeführt werden ohne ein graphische Oberfläche zu benutzen.
                \\
                \hline
                Struktur &
                \begin{itemize}
                    \itemsep0em
                    \item Alle Methoden haben einen leeren Funktionskörper
                \end{itemize}
                \\
                \hline
            \end{tabular}
        \end{table}
        \begin{figure}[H]
            \centering
            \includegraphics[scale=0.6]{img/uml/fakeGUI.png}   
            \caption{FakeGUI UML-Klassendiagramm}
        \end{figure}