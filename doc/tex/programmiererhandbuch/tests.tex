    
    Neben den automatischen Tests kann die Logik auch mit Testdateien getestet werden.
    Diese Dateien sind speziell konstruierte Szenarien in denen ein Fehler Auftritt.
    
    \begin{table}[H]
        \caption{Spielfeld Tests}
        \begin{longtable}{p{6cm} p{4cm} p{2cm}} 
            \hline
            \textbf{Testfall} & \textbf{Erwartetes Ergebnis} & \textbf{Erzieltes Ergebnis}\\
            \hline
            Es wird getestet, ob das zu ladende Spielfeld fehlerfrei ist.
            Die entsprechenden automatischen Unit-Tests (\textit{CorruptedNetwork*}) befinden sich in \textit{GameLogicTest}.
            Manuell kann dies auch mithilfe der mitgelieferten Testdateien getestet werden.
            Dazu muss jedoch das \textit{JAR} entpackt werden und die \textit{network.json} durch die
            \textit{network\_wrongJson.json} oder die \textit{network\_wrongFormat.json} oder die
            \textit{network\_wrongValues.json} ersetzt werden. Dabei ist zu beachten, dass die ersetzende Datei in \textit{network.json}
            unbenannt werden muss. Das \textit{JAR} muss zusätzlich neu gepackt werden.
            &
            Das Programm sollte eine Fehlermeldung ausgeben, dass das zu ladende Spielfeld korrupt ist.
            &
            Das Programm gibt die Fehlermeldung aus.
            \\
            \hline
        \end{longtable}
    \end{table}

    \begin{table}[H]
        \caption{Spielstand Tests}
        \begin{longtable}{p{6cm} p{4cm} p{2cm}} 
            \hline
            \textbf{Testfall} & \textbf{Erwartetes Ergebnis} & \textbf{Erzieltes Ergebnis}\\
            \hline
            \endfirsthead
            Es wird getestet, ob die Spielstandsdatei fehlerfrei ist.
            Die entsprechenden automatischen Unit-Tests (\textit{CorruptedSaveState*}) befinden sich in \textit{GameLogicTest}.
            Manuell kann dies auch mithilfe der mitgelieferten Testdateien getestet werden.
            Dazu wird die Datei \textit{save\_wrongJson.sy} geladen um eine fehlerhafte \textit{JSON}-Struktur zu testen.
            Die Datei \textit{save\_wrongFormat.sy} hingegen testet, ob die formale Struktur eingehalten wird.
            Dies bezieht fehlende Felder mit ein.
            Die \textit{save\_wrongValues.sy} testet, ob die Werte korrekt sind. Dies beinhaltet z.B. negative Ticketanzahlen.
            &
            Das Programm sollte eine Fehlermeldung ausgeben, dass der zu ladende Spielstand korrupt ist.
            &
            Das Programm gibt die Fehlermeldung aus.
            \\
            \hline
            Es wird getestet, ob ein Spielstand erfolgreich geladen wird.
            Dazu wird die \textit{save.sy} Datei geladen.
            &
            Der Spielstand lädt erfolgreich. Dabei steht Mister-X auf der Station 138 und besitzt zehn Taxiticktes,
            acht Bustickets, 4 U-Bahntickets und 2 Blacktickets. Das Fahrtenbuch ist dabei leer.
            Die Detektive stehen jeweils auf den Stationen 197, 34 und 94 und besitzen 10 Taxitickets, 8 Bustickets und 4 U-Bahntickets.
            Mister-X ist am Zug.
            &
            der Spielstand wird erfolgreich geladen und die Werte werden korrekt angezeigt.
            \\
            \hline
        \end{longtable}
    \end{table}


    \begin{table}[H]
        \caption{Interaktions Tests}
        \begin{longtable}{p{6cm} p{4cm} p{2cm}} 
            \hline
            \textbf{Testfall} & \textbf{Erwartetes Ergebnis} & \textbf{Erzieltes Ergebnis}\\
            \hline
            \textbf{Station nicht erreichbar}\\
            Es wird getestet, ob der Fehler abgefangen wird falls ein Spieler eine Station anklickt die zu weit entfernt wird.
            Dafür wird das Spiel gestartet und auf eine Station geklickt die zu weit weg ist als, dass diese erreicht werden kann.
            &
            Das Programm sollte eine Infomeldung ausgeben, dass die Station nicht in einem Zug erreicht werden kann.
            Zudem sollte der Spieler die erneute Möglichkeit haben eine Station zu wählen.
            &
            Das Programm gibt die Infomeldung aus.
            \\
            \hline
            \textbf{Station über mehrere Tickets erreichbar}\\
            Es wird getestet, ob dem Spieler eine Möglichkeit geboten wird zwischen mehreren Tickets zu wählen, falls dieser eine Station ausgewählt hat
            die über mehrere Tickets angefahren werden kann.
            Dazu klickt der Spieler auf eine erreichbare Station die über mehrere Verbindung erreichbar ist.
            &
            Dem Spieler sollte ein Dialog gezeigt werden, sodass dieser ein Ticket auswählen kann. Steuert der Spieler Mister-X wird ihm immer das Blackticket angeboten solange
            dieser noch über genügend Blacktickets verfügt.
            Falls der Spieler über kein Ticket mehr verfügt, sodass es keine Wahlmöglichekit gibt, wird die Station angefahren und das Ticket entfernt.
            &
            Dem Spieler wird der Dialog mit den richtigen Tickets angezeigt. Falls dieser Dialog abgebrochen bzw. geschlossen wird, wird das Ticket benutzt welches ausgewählt wurde.
            \\
            \hline
            \textbf{Keine Verfügbaren Tickets mehr übrig}\\
            Es wird getestet, ob dem Spieler eine Infomeldung ausgegeben wird, wenn ein Spieler versucht ein Verkehrsmittel zu benutzen
            wofür er keine Tickets mehr hat.
            Dazu wird solange gespielt, bis von einem Verkehrsmittel keine Tickets mehr vorhanden sind.
            &
            Dem Spieler sollte eine Infomeldung angezeigt werden, dass ihm keine Tickets mehr zur Verfügung stehen.
            &
            Dem Spieler wird die Infomeldung angezeigt.
            \\
            \hline
        \end{longtable}
    \end{table}

    \begin{table}[H]
        \caption{Einstellungs Tests}
        \begin{longtable}{p{6cm} p{4cm} p{2cm}} 
            \hline
            \textbf{Testfall} & \textbf{Erwartetes Ergebnis} & \textbf{Erzieltes Ergebnis}\\
            \hline
            \endfirsthead
            \textbf{Godmode}\\
            Es wird getestet, ob der Godmode funktioniert.
            Dabei wird das Spiel gestartet und auf eine Spielsituation abgewartet, in der sich Mister-X nicht zeigt.
            Nun wird auf den \textit{Godmode}-Menüpunkt gedrückt.
            &
            Mister-X sollte sich sofort zeigen. Nach wiederholten drücken auf den Menüpunkt sollte dieser sich sofort wieder verstecken.
            &
            Mister-X wird korrekt dargestellt.
            \\
            \hline
        \end{longtable}
    \end{table}

